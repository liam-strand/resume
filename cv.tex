%-------------------------------------------------------------------------------
%	DOCUMENT DEFINITION
%-------------------------------------------------------------------------------

% article class because we want to fully customize the page and not use a cv template
\documentclass[letter,10pt]{article}

%-------------------------------------------------------------------------------
%	FONT
%-------------------------------------------------------------------------------

% % fontspec allows you to use TTF/OTF fonts directly
% \usepackage{fontspec}
% \defaultfontfeatures{Ligatures=TeX}

% % modified for ShareLaTeX use
% \setmainfont[
% SmallCapsFont = Fontin-SmallCaps.otf,
% BoldFont = Fontin-Bold.otf,
% ItalicFont = Fontin-Italic.otf
% ]
% {Fontin.otf}

%-------------------------------------------------------------------------------
%	PACKAGES
%-------------------------------------------------------------------------------
\usepackage{url}
\usepackage{parskip} 	

%other packages for formatting
\RequirePackage{color}
\RequirePackage{graphicx}
\usepackage[usenames,dvipsnames]{xcolor}
\usepackage[scale=0.95]{geometry}

%tabularx environment
\usepackage{tabularx}

%for lists within experience section
\usepackage{enumitem}

% centered version of 'X' col. type
\newcolumntype{C}{>{\centering\arraybackslash}X} 

%to prevent spillover of tabular into next pages
\usepackage{supertabular}
\usepackage{tabularx}
\newlength{\fullcollw}
\setlength{\fullcollw}{0.47\textwidth}

%custom \section
\usepackage{titlesec}				
\usepackage{multicol}
\usepackage{multirow}

%CV Sections inspired by: 
%http://stefano.italians.nl/archives/26
\titleformat{\section}{\Large\scshape\raggedright}{}{0em}{}[\titlerule]
\titlespacing{\section}{0pt}{10pt}{10pt}

%for publications
\usepackage[style=authoryear,sorting=ynt, maxbibnames=2]{biblatex}

%Setup hyperref package, and colours for links
\usepackage[unicode, draft=false]{hyperref}
\definecolor{linkcolour}{rgb}{0,0.2,0.6}
\hypersetup{colorlinks,breaklinks,urlcolor=linkcolour,linkcolor=linkcolour}
\addbibresource{citations.bib}
\setlength\bibitemsep{1em}

%for social icons
\usepackage{fontawesome5}

%debug page outer frames
% \usepackage{showframe}

%-------------------------------------------------------------------------------
%	BEGIN DOCUMENT
%-------------------------------------------------------------------------------
\begin{document}

% non-numbered pages
\pagestyle{empty} 

%-------------------------------------------------------------------------------
%	TITLE
%-------------------------------------------------------------------------------

% \begin{tabularx}{\linewidth}{ @{}X X@{} }
% \huge{Your Name}\vspace{2pt} & \hfill \emoji{incoming-envelope} email@email.com \\
% \raisebox{-0.05\height}\faGithub\ username \ | \
% \raisebox{-0.00\height}\faLinkedin\ username \ | \ \raisebox{-0.05\height}\faGlobe \ mysite.com  & \hfill \emoji{calling} number
% \end{tabularx}

\begin{tabularx}{\linewidth}{@{} C @{}}
\huge{Liam Strand} \\[5pt]
\href{https://github.com/liam-strand}{\raisebox{-0.05\height}\faGithub\ liam-strand} \ $|$ \ 
\href{https://liam-strand.github.io}{\raisebox{-0.05\height}\faGlobe \ liam-strand.github.io} \ $|$ \ 
\href{mailto:strandliam@gmail.com}{\raisebox{-0.05\height}\faEnvelope \ strandliam@gmail.com} \ $|$ \ 
\href{tel:+14084381125}{\raisebox{-0.05\height}\faMobile \ +1 (408) 438-1125} \\
\end{tabularx}

%-------------------------------------------------------------------------------
% INTERESTS / KEYWORDS / SUMMARY
%-------------------------------------------------------------------------------

% \section{Interests}
% This CV is automatically generated and deployed using the \href{https://github.com/jitinnair1/autoCV}{autoCV} template along with GitHub Actions such that a new version of the CV is compiled, published and ready for use when the cv.tex file is updated. For details, \href{https://github.com/jitinnair1/autoCV}{click here}.


%-------------------------------------------------------------------------------
% EXPERIENCE
%-------------------------------------------------------------------------------

\section{Work Experience}
\begin{tabularx}{\linewidth}{ @{}l r@{} }
    \textbf{Embedded Software Engineering Intern, AeroVironment, Inc.} & \hfill Summer 2023 \\[3.75pt]
    \multicolumn{2}{@{}X@{}}{ Demonstrated that development and hardware costs of an aerial robotics product can be dramatically reduced by transitioning to COTS hardware. Created CI pipelines and containers for development using Docker and GitLab. Led porting of multiple large C++ applications to new distributed hardware environment. Refactored network communication protocol to make architectural improvements and to improve resiliency. Resolved complex Linux kernel driver compatibility issues. } \\
\end{tabularx}

\begin{tabularx}{\linewidth}{ @{}l r@{} }
    \textbf{Fowler Family Summer Scholar and Project Lead, Tufts University Computer Science} & \hfill Summer 2022 \\[3.75pt]
    \multicolumn{2}{@{}X@{}}{Led a multi-institutional team to deliver new features to the MEDFORD Language.  Features include: dynamic syntax validation, contextual color highlighting, and intelligent auto-completion. Utilized beta tester feedback and automated regression testing. Implementation uses a client/server model, with custom, well-documented API to easily support new language features and editors. Manuscript in Preparation: \textit{Context-Sensitive Editing for the MEDFORD Metadata Language} (\href{https://youtu.be/TFLdXxnaBlU}{Demo}) (\href{https://liam-strand.github.io/assets/docs/medford_poster.pdf}{Poster}) (\href{https://github.com/liam-strand/medford-language-server}{Code Repository})} \\
\end{tabularx}

\begin{tabularx}{\linewidth}{ @{}l r@{} }
    \textbf{Teaching Assistant (C, C++), Tufts University Computer Science} & \hfill Fall 2021 - Present \\[3.75pt]
    \multicolumn{2}{@{}X@{}}{Design and teach labs. Teach debugging strategies and tools (gdb, valgrind). Lead review sessions. Collaborate with fellow TAs and faculty to develop course infrastructure, assessments, and assignments for machine structure and assembly programming courses.} \\
\end{tabularx}

\begin{tabularx}{\linewidth}{ @{}l r@{} }
    \textbf{Data Collection/Analysis, Madaket Marine} & \hfill Summer/Winter 2018 - 2021 \\[3.75pt]
    \multicolumn{2}{@{}X@{}}{Designed and implemented Yamaha outboard engine data collection methodology. Gathered, analyzed, synthesized data for 18 engine models. Created accessible diagnostic charts. Recovered and transferred GPS data. Created user guide to allow replication of work.}
\end{tabularx}

% \begin{minipage}[t]{\linewidth}
%     \begin{itemize}[nosep,after=\strut, leftmargin=1em, itemsep=3pt]
%         \item[--] Developing course infrastructure
%         \item[--] Assisting in course planning and scheduling
%         \item[--] Designing and teaching labs
%         \item[--] Planning and running review sessions
%         \item[--] Hosting one-on-one and group office hours 
%     \end{itemize}
% \end{minipage}

%-------------------------------------------------------------------------------
% PROJECTS
%-------------------------------------------------------------------------------

\section{Personal Projects}

\begin{tabularx}{\linewidth}{ @{}l r@{} }
    \textbf{Rust Boggle Solver: \href{https://github.com/liam-strand/ruggle}{Ruggle}} & \hfill 2023 \\[3.75pt]
    \multicolumn{2}{@{}X@{}}{Designed and built a highly optimized boggle solver in Rust utilizing a serializable prefix tree (trie) data structure to encode a lexicon, a pruning graph-traversal algorithm, and a custom visualization system to display words as they appear on the board.}  \\[3.75pt]
\end{tabularx}

\begin{tabularx}{\linewidth}{ @{}l r@{} }
    \textbf{Assembly Emulator:} \href{https://github.com/liam-strand/comp-40-VM-Profiling}{\textbf{VM Profiling}} & \hfill 2022 \\[3.75pt]
    \multicolumn{2}{@{}X@{}}{Built a simulated architecture emulator in C, and vastly improved performance using modern profiling techniques. (\href{https://youtu.be/OnzkSmFvxiM}{Demo})} \\[3.75pt]
\end{tabularx}

\begin{tabularx}{\linewidth}{ @{}l r@{} }
    \textbf{Concurrent Simulations:} \href{https://github.com/liam-strand/cs-21-final-project}{\textbf{on-the-road-again}} and \href{https://github.com/liam-strand/cs121-p5\#metrosim2}{\textbf{MetroSim2}} & \hfill 2022 \\[3.75pt]
    \multicolumn{2}{@{}X@{}}{Developed traffic simulation using CSP in Erlang with Python visualization, communicating over serialized I/O. Developed a fully concurrent subway simulation in Java using communicating threads to represent trains, passengers, and stations. (\href{https://youtu.be/_JQW8edL14A}{Demo})}  \\[3.75pt]
\end{tabularx}

\begin{tabularx}{\linewidth}{ @{}l r@{} }
    \href{https://github.com/liam-strand/cs121-p3\#testing-in-java}{\textbf{Java Testing Framework}} & \hfill 2022 \\[3.75pt]
    \multicolumn{2}{@{}X@{}}{Designed testing infrastructure: unit testing library, test generation framework, fluent assertion domain-specific language, test runner.}  \\[3.75pt]
\end{tabularx}

\begin{tabularx}{\linewidth}{ @{}l r@{} }
    \textbf{Python CLI Concurrent Queue Manager:} \href{https://github.com/liam-strand/HalliganHelper2}{\textbf{HalliganHelper2}} & \hfill 2021 \\[3.75pt]
    \multicolumn{2}{@{}X@{}}{Developed CLI to manage FIFO queue of students requesting TA assistance in Python. Used shared files and file locks to prevent change conflicts. CLI automatically invokes relevant commands and imposes serialization. (\href{https://youtu.be/qPwZz3OVA7A}{Demo})} \\[3.75pt]
\end{tabularx}

% \begin{tabularx}{\linewidth}{ @{}l r@{} }
%     \textbf{iOS Data Scraping App:} \href{https://github.com/liam-strand/IFR-Plate-Scrape}{\textbf{IFR Plate Scrape}} & \hfill 2021 \\[3.75pt]
%     \multicolumn{2}{@{}X@{}}{Built airport mapping app in Python using HTTP requests to retrieve FAA data, and BeautifulSoup to extract and present relevant information to user. Ported to iOS using Swift, leveraging built-in URL management and SwiftSoup. (\href{https://youtu.be/bDe3dsVyUok}{Demo})} \\[3.75pt]
% \end{tabularx}

\begin{tabularx}{\linewidth}{ @{}l r@{} }
    \textbf{iOS Data Logging App:} \href{https://github.com/bstrand42/Log-Your-Catch}{\textbf{Log Your Catch}} & \hfill 2020 \\[3.75pt]
    \multicolumn{2}{@{}X@{}}{Built iOS app to capture and record data to support fishery management using Apple Location Services and Google Firebase database services. Completed architecture design, UI design, system integration, authentication, and beta testing.} \\[3.75pt]
\end{tabularx}

%-------------------------------------------------------------------------------
%	SKILLS
%-------------------------------------------------------------------------------
\section{Skills}
\normalsize{C, C++, Concurrent/Parallel Programming, Profiling, Performance Analysis, Tuning, Linux/Unix, Testing Methodologies, Python, Rust, Java, Docker, \texttt{git}, GitHub, GitLab, Jira, Agile Development, \texttt{gdb}, Erlang, Swift, \LaTeX}

%-------------------------------------------------------------------------------
%	EDUCATION
%-------------------------------------------------------------------------------
\section{Education}
\begin{tabularx}{\linewidth}{@{}l X@{}}	
2020 - Present & BS in Computer Science, Tufts University School of Engineering (ABET) \hfill \normalsize GPA: 3.96/4.00 \\
    & \footnotesize{Data Structures/Algorithms, Machine Structure and Assembly Language, Software Engineering, Security, Concurrent Programming, Parallel Computing, Computational Theory, Programming Languages, Fall 23: Machine Learning} (Dean's List 6/6 semesters) \\[3.75pt]
Summer/Fall 2020 & Swift 2, iOS development bootcamp, Udemy London App Brewery
% Summer 2019 & Python Scripting, Rice University/Coursera	
\end{tabularx}

%-------------------------------------------------------------------------------
%	PUBLICATIONS
%-------------------------------------------------------------------------------
% \section{Publications}
% \begin{refsection}[citations.bib]
% \nocite{*}
% \printbibliography[heading=none]
% \end{refsection}


% \vfill
% \center{\footnotesize Last updated: \today}

\end{document}
